%  Proposal for checking the DRAGON energy calibration with ref. to 27Al(p,g) at 203keV
\documentclass[12pt]{amsart}
\usepackage{geometry} % see geometry.pdf on how to lay out the page. There's lots.
\geometry{a4paper} % or letter or a5paper or ... etc
% \geometry{landscape} % rotated page geometry

% See the ``Article customise'' template for come common customisations

%\title{Branching}
%\author{D.A. Hutcheon}
%\date{} % delete this line to display the current date

%%% BEGIN DOCUMENT
%\begin{document}

%\maketitle
%\tableofcontents

%\section{}
%\subsection{}

%\end{document}
%  
%\documentclass[12pt]{article}
\usepackage{graphicx}
\setlength{\textwidth}{15.5cm}
\setlength{\oddsidemargin}{.50cm}
\setlength{\evensidemargin}{.50cm}
\setlength{\topmargin}{-1.cm}
\setlength{\textheight}{22.cm}
\thispagestyle{empty}
\def\insertplot#1#2#3{\par
 \hbox{\hskip #3 \vbox to #2{\special{hp: plotfile #1}\vfil}}}

\begin{document}
%\setlength{\baselineskip}{4ex}
%\frenchspacing
%\maketitle
\begin{center}
{\Large \bf Separators for Radiative Capture with Inverse Kinematics  }\\
\vspace{0.2cm}
%{\em DRAGON Note} \\
D.A. Hutcheon \\
\today
\end{center}

\section{Electromagnetic separators}
The heavy-ion products of  inverse-kinematics capture reactions emerge in nearly the same direction as the non-reacting beam ions, but with an intensity that may be  lower by ten or more orders of magnitude.   In consequence, any detector positioned to intercept directly the heavy reaction product will be overwhelmed by beam ions.   A device is required which can  suppress almost all of the beam before it can reach the detector: the device of choice is an electromagnetic separator operated at high vacuum.

The masses of desired product ions and beam ions may differ by a few percent for proton capture to as much as 25 percent for alpha capture by a light nucleus.   The average momenta of the two species being essentially the same, the corresponding differences in kinetic energy and velocity provide a basis for electromagnetic (EM) separation.    A complicating factor is the range  of several percent in energies of desired product,  which results from the impulse imparted by the capture gamma rays.    In addition, ions having the same velocity and the same mass-to-charge ratio will follow identical trajectories through the EM fields: high-vacuum mass separators (as distinct from gas-filled separators) are ``mass-over-charge" separators.

\subsection{Magnetic dipoles}
An ion moving in a uniform magnetic field follows a circular path, the radius of which is proportional to its  momentum.  The magnetic rigidity, the momentum $p$ divided by the ion charge $q$, is proportional to the product of $\rho$ the radius of curvature and $B$ the magnetic field, so ions of differing magnetic rigidity will be deflected through differing bend angles  when passing through a dipole magnet.  The reaction products and beam ions having overlapping values of momentum, a magnetic dipole by itself is ineffective as a mass separator.  However, it can be used to separate ions in a low-energy tail of the beam, which might be transmitted by electric field devices.

\subsection{Electrostatic dipoles}
A cathode and anode in the form of segments of concentric cylinders can produce a radial electric field which has constant magnitude $\mathcal{E}$  at a constant distance from the axis of the cylinders.   Ions having a suitable electric rigidity, the product of momentum $p$ and velocity $v$ divided by the charge $q$, can travel at a constant radius through the device.   Such an electrostatic dipole will deflect reaction products and beam ions through different angles due to their different velocities.  However, a low-energy tail of the beam, in a lower charge state and having the same electric rigidity as the reaction product, will not be separated.   This short-coming can be remedied by combining a magnetic dipole and electrostatic dipole in such a way that a subsequent angular focus is also a focus in ion energy but is dispersed according to ion mass.  

The required electrostatic fields are of order megavolts per metre.   Extensive measures, such as polishing the electrodes to a mirror finish, are required to prevent voltage breakdowns.   Because an electrostatic dipole must deflect desired particles through a permanently-fixed angle,  an experiment  cannot be performed unless the requisite   electrode voltages can be achieved.  

\subsection{Wien filters}
In a Wien filter parallel flat electrodes between the poles of a magnet provide crossed electric and magnetic fields, $\mathcal{E}$ and $B$.      There is a certain ratio $\mathcal{E}/B$ which results in ions of velocity $v$  undergoing no deflection as they pass through the Wien filter, regardless of their momentum $p$ or charge $q$.    Accordingly, a Wien filter can be tuned to pass the desired reaction product straight through while deflecting beam ions.  Like the electrostatic dipole, it may pass beam particles in  a low-energy tail; unlike the electrostatic dipole, it  also allows   neutral particles to pass through.

Because it is  the ratio $\mathcal{E}/B$ which selects the velocity, a Wien filter may operate with reduced\ field strengths (and reduced mass resolving power)  in order to accommodate reaction products of high rigidity, or if there is a problem reaching the designed maximum electric field.    On the other hand, a Wien filter poses greater design problems, requiring careful shaping of both the electrodes and the magnet poles in order to achieve the desired field uniformities and to match fringe fields.

\subsection{Focusing and aberration correction}
Quadrupole magnets provide focusing and control over the envelope of trajectories.  For example, it may be desired to produce an achromatic focus following a magnetic dipole and electrostatic device, which requires, in addition to the spatial focus, an appropriate relationship between the trajectory envelopes through the dispersive devices.    In second order, the quadrupoles and dispersive elements may introduce spherical or chromatic aberrations.    Sextupole magnets can provide correction of second-order aberrations.   Alternatively, corrections may be made by shaping of dipole entrance and exit field boundaries   or shaping poles to create a sextupole component in a quadrupole.   These alternative measures allow for a more compact separator having fewer discrete elements, but  are not readily tunable.

\section{Design considerations}
\subsection{Beam suppression }
Reaction yield and detector performance together define the requirements for beam suppression by the separator.    The most challenging reactions  may well call for suppression by 10 to 12 orders of magnitude.    Separator designs for fusion-evaporation reactions, with higher yields and larger beam-product mass differences, may not be the best models  for radiative capture experiments.   
    
 One class of potential background is the tail of a  distribution generated by many small interactions, such as multiple scattering or energy straggling in the target.    Typically, these distributions have a Gaussian shape near their maximum but after only a few  ``standard deviations" the distributions become dominated by single large-deviation interactions, such as nuclear scattering in the target.    More complicated to assess are single scatterings from residual gas somewhere within the separator.   Nevertheless, it may be possible to investigate by computer simulation whether  a proposed separator design is able to suppress such events.
 
  ``Black Swan" events pose the most difficult problem for the separator designer.   By definition extremely rare and hard to foresee,  they are not  amenable to realistic simulation by computer.   Typically, they may involve sequential scatterings from  solid surfaces inside the separator.   The ranges of beam ions in solids being on order 1--10~$\mu$m, the surface material and roughness at the micrometer scale play a big role: oblique incidence on a smooth surface should be avoided and if possible such surfaces should be masked.   The separator designer must look to defense in depth, with more beam suppression features than the minimum required for events of the first  class.
  
  \subsection{Figure of merit}
  Some specifications, taken in combination, impose minimum requirements on the strength and extent of electric and magnetic fields.   These are the mass resolving power ($P_\mathrm{m}$), the angular acceptance ($\Delta a$) and the maximum particle rigidity 
  ($\mathcal{R}_\mathrm{M}=p/q$ or $\mathcal{R}_\mathrm{E}=pv/q$).    
  Consider a source of ions (reactions in the target), a dispersive device  (radius of curvature $\rho$) and an image (focus) point.   The  trajectories of the ions having the extremes in initial angles will delimit an area ($A$)  in the bend-plane of the dispersive device.     To first order the resolving power is given by the linear magnification ($M$) of the image, the mass dispersion ($D$) and the transverse source size ($\Delta x$): 
  $P_\mathrm{m} = D/(M \cdot \Delta x) $. 
  It has long been known (see, for example, \cite{Wo71}) that source emittance and the resolving power can be related to properties of the dispersive device by
  \[ P_\mathrm{m}\  \Delta a\  \Delta x = A/\rho \]
  Recalling that 
  $\mathcal{R}_\mathrm{M}=\rho \ B$ and $\mathcal{R}_\mathrm{E}=\rho \ \mathcal{E}$,  we obtain  figures of merit 
  \[ P_\mathrm{m}\  \Delta a\  \Delta x \ \mathcal{R}_\mathrm{M} = A \cdot B \]
   \[P_\mathrm{m}\  \Delta a\  \Delta x \ \mathcal{R}_\mathrm{E} = A \cdot \mathcal{E} \]
   That is, to achieve a certain figure of merit it is necessary to have a certain minimum product of field strength and area.  For an electrostatic device it may be convenient to use the length of electrodes and the gap between them as a proxy for the area $A$.  Then, the figure of merit becomes $L\cdot\Delta V$, where $L$ is the length of the electrodes and $\Delta V$ is the voltage difference between them.

\subsection{Additional factors} 
  Windowless targets either extended gas cell or gas jet, can impose major constraints on separator design.  Tubes between differential pumping stages may limit the convergence of incoming beam and consequently limit how small the beamspot  can be made at the target.  Pumping stages may set a minimum distance from target to the first focusing element.

Detectors of the heavy reaction product may have a maximum counting rate for optimum resolution, requiring a matching minimum beam suppression factor for the separator.   The detector may be limited in size, requiring a correspondingly small envelope of trajectories.    Good energy resolution or particle identification in a detector may permit a higher rate of beam leakage through the separator.

 The space   for the separator may be limited or the wrong shape if the experimental area was designed for the needs of earlier experiments.   Dipoles, quadrupoles  or Wien filters from elsewhere  may be available for use.  Finally,  funds for the project may be limited.   

\section{Some existing separators for radiative capture}
\subsection{ERNA}
The European Recoil separator for Nuclear Astrophysics, at Bochum, was designed with a view to measurement of the $^{12}$C($\alpha$,$\gamma$)$^{16}$O reaction in inverse 
kinematics~\cite{Ro99}.  The dispersive elements consist of a Wien filter, a 60-degree magnetic dipole and a second Wien filter.    With electrode voltages up to $\pm$50~kV and lengths of 0.50 and 0.578~m, the Wien filters have a proxy figure of merit 108~kV$\cdot$m.   The design envisaged an energy range 0.7 to 5~MeV and an angular acceptance 2$\times$31~mrad.

\subsection{DRS}
The Daresbury Recoil Separator~\cite{Ja88}  was built and used at the Daresbury Laboratory, before being moved to Oak Ridge National Laboratory.  Dispersive elements are a pair of Wien filters followed by a 50-degree magnetic dipole.   With maximum design voltage $\Delta$V=600~kV and each of length 1.28~m, the two Wien filters have a proxy figure of merit 1536~kV$\cdot$m.
Design acceptances were 2$\times$45~mrad in horizontal and vertical angles, $\pm$2\% in velocity and  $\pm$1.2\% in M/q, with resolving power 300 in M/q.
 
\subsection{DRAGON}
The Detector of Recoils And Gammas of Nuclear reactions at TRIUMF/ISAC was designed for proton and alpha capture by radioactive beams~\cite{hut03b}.  It consists of two stages, each containing a magnetic dipole and electrostatic dipole.   The electrostatic dipoles, with respective lengths 0.7 and 1.57~m and voltages $\pm$200 and $\pm$160~kV, give a proxy figure of merit 780~kV$\cdot$m.  Design angular acceptance was 2$\times$21~mrad with beam energies in the range 0.15 to 1.5~MeV per nucleon.

\subsection{FMA}
The Fragment Mass Analyzer at Argonne National Laboratory~\cite{Da92} was designed for study of high-spin states following heavy-ion fusion.  Like mass separators at Rochester and Legnaro, the FMA has a series of electric, magnetic and electric dipoles.   At $\Delta$V=425~kV and each of length 1.4~m, the two electrostatic dipoles have proxy figure of merit 1260~kV$\cdot$m.  Design acceptances were $\pm$15\% in energy and 8~mster solid angle, with M/q resolving power of 340.


\begin{thebibliography}{99}
\bibitem{Wo71} H. Wollnik, Nuclear Instruments and Methods {\bf 95}, 453 (1971).
\bibitem{Ro99} D. Rogalla {\em et al.}, Eur. J. Phys. A {\bf 6} 471 (1999).
\bibitem{Ja88} A.N. James {\em et al.}, Nucl. Instr. and Methods {\bf A267} 144 (1988).
\bibitem{Hu03} D.A. Hutcheon {\em et al.}, Nucl. Instr. and Methods {\bf A498} 190 (2003).
\bibitem{Da92} C.N. Davids {\em et al.},  Nucl. Instr. and Methods {\bf B70} 358 (1992)
\end{thebibliography}
 \end{document}

