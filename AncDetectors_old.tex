\section{Ancillary detectors and methods}

\subsection{Beam normalisation}

All measurements of cross sections or resonance strengths require knowledge of the number of particles incident on the target number areal density. Traditionally, with the setup in normal kinematics, this is accomplished by measuring and integrating the ion beam current continuously. Alternatively, in previous inverse kinematics experiments with windowless gas targets, the thermal beam power deposited after traversing the target gas was measured in a beam calorimeter. This approach was necessary as the charge exchange in the gas passage is strongly dependent on gas pressure and beam type and energy, making a Faraday cup charge integration difficult to interpret. Another alternative pursued in the past in gas target experiments is the beam current normalization using elastic scattering of the beam ion on the target gas, which requires precise knowledge of the cross section in question. The use of recoil separators in our experiments deprives us of the possibility to pursue the first two methods as they require blockage of the ion (and recoil) beam path after the target. However, as the recoil separators purpose is to separate out the incoming ion beam, beam normalization (or diagnostic) devices can be designed to intercept all (or part, {\it e.g.} one charge state) at a location where recoils are transmitted and the beam (due to difference in mass-to-charge ratio, velocity or energy) is deposited on a selective slit. Using radioactive ion beams allows the experimenter also to monitor the radioactive decays at the locations where beam is deposited in the separator. Care has to be taken, however, that different beam focussing does not affect detection efficiencies (through different deposit size or location). Additionally, the use of temporary beam intercepts is always possible, monitoring either charge or scattering in the moment of intercept, or radioactive decay of the material deposited in the non-intercept position.

\subsection{Prompt $\gamma$-ray detection}

\small
\begin{itemize}
\item aids and gives additional information to recoil detection (ToF, tagging, different g-transitions)
\item needs high efficiency due to low yields expected in nuclear astrophysics experiments
\item sufficient resolution to allow for threshold setting excluding 511 keV due to beta+ decay of deposited RIB beam
\item aids also in focussing through gas target flow limiting apertures (minimize RIB deposits)
\item efficiency/price considerations usually discourage the use of Germanium arrays (alternatives NaI, BGO, BaF2...)
\end{itemize}
\normalsize

\subsection{Local time-of-flight}

\cite{vock09}



\subsection{Silicon strip detectors}

\small
\begin{itemize}
\item DSSSDs
\item monolithic silicon detectors
\end{itemize}
\normalsize


\subsection{Ionization chambers}

